\section{Objetivos}

\subsection{Objetivo geral:}

O objetivo do presente estudo será aplicar modelos de regressão linear para analisar o desempenho de corredores de \textit{skyrunnig} utilizando dados do Strava.

\subsection{Objetivos específicos:}

\begin{enumerate}[1.]

\item Utilizar técnicas de webscraping através da linguagem de programação python para coletar e preparar uma base de dados, contendo informações como tempo, distância, altitude, ritmo, entre outros.

\item Explorar e analisar os dados coletados do Strava para identificar padrões e tendências no desempenho dos corredores, diferenças entre corredores de diferentes níveis de habilidade, entre outros aspectos.

\item Aplicar técnicas de regressão linear para modelar a relação entre variáveis independentes (como desnível, pacing, distância, frequência cardíaca) e dependente (desempenho do corredor), explorando como elas podem influenciar o desempenho dos corredores.

\item Interpretar os resultados da análise de regressão linear para identificar os principais fatores que influenciam o desempenho dos corredores.

\item Auxiliar assessorias esportivas e/ou corredores amadores a traçar estratégias para melhorarem o desempenho em uma prova de \textit{skyrunnig}.
\end{enumerate}