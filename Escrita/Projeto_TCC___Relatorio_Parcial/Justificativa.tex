\section{Justificativa}

Ao observar a lacuna existente na exploração acadêmica ao que se diz respeito a aplicação de técnicas estatísticas envolvendo temas relacionados ao desempenho de atletas, notou-se a viabilidade de elaborar um projeto de pesquisa com ênfase em corredores, sendo eles amadores ou profissionais. 

Entender o desempenho dos corredores é fundamental porque fornece ideias valiosas não apenas para os próprios atletas, mas também para treinadores, profissionais de saúde esportiva e até mesmo pesquisadores da área. Ao compreender as variáveis que afetam o desempenho, como ritmo, desnível do percurso, entre outras, podemos identificar estratégias mais eficazes de treinamento e competição, maximizando o potencial de cada corredor.

Além disso, analisar essas variáveis não apenas fornece uma compreensão mais profunda do desempenho individual, mas também contribui para o desenvolvimento de métodos de treinamento mais personalizados e eficazes. Por exemplo, um corredor pode se beneficiar de ajustes em sua estratégia de pacing com base em suas características físicas e objetivos de desempenho. Da mesma forma, conhecer o impacto do desnível do percurso pode ajudar os atletas a planejar suas estratégias de corrida de forma mais eficiente, adaptando-se melhor às condições específicas da prova.

Quanto ao uso da regressão linear, essa técnica estatística é uma ferramenta poderosa para modelar e compreender as relações entre variáveis independentes e dependentes. No contexto da análise do desempenho dos corredores, a regressão linear nos permite quantificar o impacto de cada variável sobre o desempenho geral do corredor, fornecendo uma base sólida para a tomada de decisões informadas no treinamento e na competição.

Portanto, utilizar a regressão linear neste estudo não apenas nos permite explorar essas relações de forma mais precisa, mas também nos ajuda a identificar padrões e tendências nos dados, fornecendo insights valiosos que podem ser aplicados de maneira prática para melhorar o desempenho e a saúde dos corredores. Assim, este estudo não só contribui para o avanço do conhecimento científico na área, mas também tem o potencial de impactar positivamente a comunidade de corredores, auxiliando na melhoria contínua de suas práticas de treinamento e competição.
