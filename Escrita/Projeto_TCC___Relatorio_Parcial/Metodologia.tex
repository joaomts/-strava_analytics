\section{Metodologia}

 Neste trabalho de conclusão de curso será realizado um estudo descritivo. Iniciando-se pela extração dos dados de uma prova específica de \textit{skyrunnig} da plataforma de rastreamento de exercícios físicos Strava, através de técnicas de webscraping com a linguagem de programação python. 
 
 A população alvo do estudo consiste em corredores maratonistas de provas de skyrunning. Os dados coletados serão armazenados e organizados em um arquivo Comma Separated Values(CSV), em que as variáveis serão agrupadas por cada corredor participante da prova. Os participantes não serão identificados no conjunto. 

 A análise dos dados coletados se dará por meios de técnicas de estatísticas descritivas buscando compreender características e comportamentos dos corredores da amostra extraída, assim como identificar perfis de corredores através de técnicas de agrupamentos e detecção de perfil. 
 
 Ao ajustar o modelo de regressão linear multivariada utilizando como características o pacing, desnível, frequência cardíaca, entre outras, como variáveis independentes, será possível prever o tempo final de prova do atleta. Tendo como objetivo determinar fatores que influenciarão no desempenho do corredor.

 Para finalizar, serão aplicadas técnicas de avaliação de modelos de regressão linear por meio da análises de resíduos e testes de significância dos coeficientes estimados buscando, assim, entender a relação entre as variáveis independentes e o desempenho dos corredores.


 


 
 
 
