\section{Resultados Parciais}

Ao fim do primeiro semestre do Trabalho de Conclusão de Curso, foi realizado de forma integral os objetivos estabelecidos pra esta primeira parte do projeto.

Os dados foram coletados com exito, ulizando-se de técnicas de WebScraping em conjunto com a linguagem Python \cite{mitchell2019}. Ao fim coletamos dados de duas provas de skyrunning com um percurso total de 35Km, chamada 'La Mision Brasil' a prova realizada na cidade de Passa Quatro-MG, referente aos anos de 2022 e 2023. Foram obtidos dados de 109 corredores na etapa de 2022 e 171 corredores em 2023. Coletamos para cada atleta variáveis como ritmo médio e elevação de cada quilômetro percorrido do atleta.
Os dados obtidos foram organizados em um único conjunto de dados e salvo em um arquivo CSV. 

A fim de agregar mais valor às análises futuras e à construção do modelo de regressão linear, planejamos executar uma segunda fase de webscraping. Esta etapa adicional de coleta de dados visa obter uma gama mais abrangente de variáveis que podem influenciar o desempenho dos corredores em provas de skyrunning.

A inclusão de variáveis como idade, peso e batimentos cardíacos por minuto (bpm) é fundamental para uma análise mais holística do desempenho atlético. Por exemplo, a idade pode afetar a resistência e a capacidade de recuperação do atleta, enquanto o peso e os batimentos cardíacos por minuto podem indicar o nível de condicionamento físico e a intensidade do esforço durante a corrida.

Para realizar essa fase adicional de coleta de dados, continuaremos a utilizar técnicas de webscraping em conjunto com a linguagem de programação Python, as mesmas utilizadas na fase anterior. No entanto, reconhecemos que podem surgir desafios adicionais, como a disponibilidade das informações desejadas na web e a qualidade dos dados coletados. Portanto, estaremos atentos a esses aspectos e faremos os ajustes necessários no processo de coleta.

Com a conclusão recente da fase de coleta de dados, devido à complexidade do processo, ainda não foi possível realizar uma análise exploratória detalhada. No momento, dispomos apenas de informações iniciais sobre os dados obtidos. Por exemplo, na edição de 2022, registramos a participação de 171 atletas que concluíram a prova, enquanto em 2023 esse número foi menor, com 109 atletas completando o percurso. Vale destacar que nossa amostra apresenta uma composição diferenciada, com aproximadamente 31\% de participação feminina, enquanto os homens representam os restantes 69\%.

Para a próxima etapa, estamos planejando uma análise exploratória mais aprofundada dos dados, com o objetivo de obter uma compreensão mais completa do conjunto de informações que temos em mãos. Isso nos permitirá direcionar nossa pesquisa de forma mais assertiva. Com um entendimento sólido dos dados coletados, iniciaremos a modelagem estatística, buscando ajustar um modelo de regressão linear. Nosso objetivo é identificar as variáveis que impactam o tempo final de conclusão da prova pelo atleta e entender como essas variáveis influenciam esse resultado. Essa abordagem nos ajudará a ganhar insights valiosos sobre os fatores que afetam o desempenho dos corredores e a otimizar nossas análises e conclusões futuras.







