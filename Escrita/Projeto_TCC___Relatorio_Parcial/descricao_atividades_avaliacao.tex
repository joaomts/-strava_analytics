\section{Descrição das Atividades e Avaliação}
\subsection{Descrição das Atividades}
\subsubsection{Atividades previstas e já desenvolvidas:}

As atividades previstas e já desenvolvidas incluem a coleta de dados por meio de técnicas de webscraping em conjunto com a linguagem Python, conforme descrito na seção de Resultados Parciais. Essa etapa envolveu a obtenção de dados de duas provas de skyrunning, 'La Mision Brasil', realizadas nos anos de 2022 e 2023, com um percurso total de 35km. Os dados foram coletados para cada atleta participante, incluindo variáveis como ritmo médio e elevação de cada quilômetro percorrido. Além disso, organizamos esses dados em um único conjunto de dados e os salvamos em um arquivo CSV, prontos para análise futura.

\subsubsection{Atividades previstas e que não serão desenvolvidas:}

Até o momento, não identificamos atividades previstas no projeto que não serão desenvolvidas. Todas as atividades planejadas foram realizadas conforme o cronograma estabelecido. No entanto, caso surjam mudanças no escopo do projeto que levem à exclusão de alguma atividade, estaremos preparados para revisar e atualizar esta seção de acordo.

\subsubsection{Atividades não previstas e que foram ou serão desenvolvidas:}

Durante a fase inicial do projeto, identificamos a necessidade de aprofundar a coleta de dados, especialmente com relação às características dos atletas participantes das provas de skyrunning. Portanto, planejamos executar uma segunda fase de webscraping para obter informações adicionais, como idade, peso e batimentos cardíacos por minuto (bpm). Essas variáveis foram consideradas importantes para uma análise mais abrangente do desempenho atlético e para a construção do modelo de regressão linear. Estamos comprometidos em garantir a integridade e a qualidade dos dados coletados, mesmo que isso exija atividades adicionais não previstas inicialmente.

\subsection{Autoavaliação do Aluno}

Na primeira parte do desenvolvimento do projeto, sinto que consegui atingir a maioria dos objetivos estabelecidos, muitas vezes quando surgiu problemas e dúvidas consegui por conta própria encontrar soluções e respostas, outrs vezes pude contar com ajuda do meu orientador para sanar algumas dúvidas e me mostrar o caminho. 

Em relação ao resultado do trabalho até o momento, estou satisfeito com o progresso alcançado. A coleta de dados foi concluída com sucesso, e os resultados preliminares indicam que temos uma quantidade significativa de informações para análise. No entanto, reconheço que ainda há espaço para aprimoramento na análise exploratória dos dados e na identificação de padrões ou insights adicionais.

Para a segunda parte do projeto, minhas expectativas são de realizar uma análise mais detalhada dos dados coletados e desenvolver um modelo de regressão linear robusto, para entender melhor o que influencia no desempenho dos corredores em provas de skyrunning. Além disso, planejo realizar uma revisão crítica dos resultados obtidos e explorar possíveis melhorias.

Estou confiante de que, com dedicação e foco, poderei contribuir significativamente para o avanço do projeto na segunda parte e alcançar resultados ainda mais satisfatórios.

\vspace{0.5cm} % Espaço de 2cm entre o texto e a assinatura

\begin{center}
Assinatura do aluno: \underline{\hspace{8cm}} % Assinatura centralizada com 8cm de comprimento
\end{center}


\subsection{Avaliação do Orientador}

O aluno João Marcos de Souza Matos cumpriu com êxito todas as atividades que haviam sido planejadas nesta primeira etapa do trabalho. Nenhuma atividade deixou de ser executada e o aluno foi capaz de propor novas abordagens durante o estudo. Com entusiamos e dedicação, o aluno tem desempenhado um ótimo trabalho e estou certo que este desempenho será mantido nas próximas etapas de nosso trabalho.




\vspace{1cm} % Espaço de 2cm entre o texto e a assinatura
\begin{center}
Assinatura do orientador: \underline{\hspace{8cm}} % Assinatura centralizada com 8cm de comprimento
\end{center}