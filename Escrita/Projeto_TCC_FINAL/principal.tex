    %====================================================================
    % Universidade Federal de Uberlândia
    % Faculdade de Matemática
    % Curso: Bacharelado em Estatística
    % Modelo para elaboração de monografia referentes aos trabalhos de
    % conclusão de curso do Bacharelado em Estatística
    % Autor: Prof. Dr. Alessandro Alves Santana
    %====================================================================
    \documentclass[12pt,openright]{report}
    %====================================================================
    % Pacote de estilo que tem por finalidade carregar pacotes do LaTeX
    % bem como definir configurações do texto da monografia. 
    %====================================================================
    \usepackage{estilo}
    %====================================================================
    % Inserção do arquivo com a lista de abreviaturas e símbolos.
    %====================================================================
    \makenomenclature
    %====================================================================
    % Documento Principal
    %====================================================================
    \begin{document}
    \tolerance=5000
    %====================================================================
    % Para que as linhas do texto sejam numeradas com a finalidade 
    % de facilitar a correção pelo orientador, basta descomentar o
    % o comando abaixo. Esse comando faz com que o arquivo pdf gerado 
    % após a compilação sem erros tenham as linhas da monografia sejam
    % enumeradas. Com as linhas numeradas o orientador poderá localizar 
    % e descrever melhor para os orientandos os locais do texto que 
    % exigem correção.
    %====================================================================
    % \linenumbers

    %=================================================================

    %====================================================================
    % O comando \capa tem 6 argumentos de entrada os quais são 
    % descritos abaixo. Tem por função gerar a capa da monografia. 
    %
    % \capa
    % {título da monografia}
    % {nome do orientando}
    % {ano de defesa da monografia}
    % {nome do orientador}
    % {nome do segundo membro da banca}
    % {nome do terceiro membro da banca}
    %
    % Observações: 
    %
    %  1- Escrever o título do trabalho em letras maíusculas.
    %       Exemplo: ANÁLISE DE REGRESSÃO EM PROBLEMAS DE FÍSICA
    %  2- Nome do aluno com iniciais em letras maiúsculas.
    %       Exemplo: Alberto Barbosa da Cunha
    %  3- Ano da defesa: 
    %       Exemplo: 2018
    %  4- Nome do Orientador e demais membros da banca de defesa: 
    %     Escrever o nome apenas com as iniciais em letras maiúsculas 
    %     como no nome do aluno.
    %       Exemplo: João Carlos Silva de Almeida    
    %====================================================================
    \capa
    {ANÁLISE DE DESEMPENHO DE CORREDORES UTILIZANDO DADOS DO STRAVA: UMA ABORDAGEM DE REGRESSÃO LINEAR.}
    {João Marcos de Souza Matos}
    {2025}
    {Pedro Franklin Cardoso Silva}
    %{Nome completo do segundo membro da banca de defesa}
    %{Nome completo do terceiro membro da banca de defesa}
    %====================================================================
    % Agradecimentos: Os agradecimentos deverão ser escritos dentro do 
    % arquivo agradecimentos.tex. 
    %====================================================================     
    \pagenumbering{Roman} % Inicia a contagem em algarismos romanos
    
    
\chapter*{Agradecimentos}
\thispagestyle{empty}


Nesta seção, desejo expressar minha profunda gratidão a todos que, de formas distintas, contribuíram para a realização deste trabalho e para minha formação acadêmica e pessoal.

Primeiramente, agradeço a Deus por me dar forças para continuar e não desistir diante das dificuldades, me proporcionando sabedoria e paciência em momentos desafiadores.

Ao meu orientador, Prof. Pedro Franklin Cardoso Silva, agradeço pela orientação, paciência e pela sua constante prontidão em me ajudar. Sua dedicação e conhecimento são inspiração para mim.

À Universidade Federal de Uberlândia e ao Instituto de Matemática e Estatística, por oferecerem um ambiente acadêmico rico e estimulante, que foi fundamental para meu crescimento intelectual.

Aos membros da banca examinadora, por aceitarem o convite de avaliar este trabalho. O tempo e conhecimento de vocês serão de grande valia para o aprimoramento deste estudo.

À Bianca Souza, que sempre esteve ao meu lado para me incentivar nos momentos de dúvida, para me erguer nas quedas e para celebrar cada conquista comigo. Você é um dos pilares mais importantes da minha vida.

Aos meus pais, Fernanda de Souza e Marco Matos, por todo amor, colo, amparo e apoio incondicional. Vocês são minha base e meu exemplo de vida.

Aos meus irmãos, Vitória, José e Antônio, por serem minha fonte de alegria.

Aos meus amigos e colegas de curso, por compartilharem essa jornada comigo, tornando-a mais leve e divertida. Alguns de vocês se tornaram verdadeiros irmãos para mim.
    \cleardoublepage
    %====================================================================
    % Resumo em português: O resumo em português deverá ser escrito 
    % dentro do arquivo resumo.tex.
    %====================================================================
    % Observação: O arquivo resumo.tex já existe dentro da 
    % pasta monografia. O que o aluno precisa fazer é abrí-lo 
    % e preenchê-lo com o texto do resumo. Não esqueça de 
    % de colocar as palavras-chave em português. No arquivo 
    % resumo.tex isso poderá ser observado. 
    %====================================================================
    %\chapter*{Resumo}
\thispagestyle{empty}
\label{chap:resumo}

\textbf{Introdução:} A corrida de montanha tem ganhado destaque entre modalidades esportivas ao ar livre, sendo influenciada por fatores como: ritmo, desnível e inclinação. Dados de plataformas de rastreamento de atividades permitem analisar variáveis de desempenho, oferecendo uma base para compreender melhor como esses elementos impactam os resultados dos corredores.
\textbf{Objetivo:} Aplicar modelos de regressão linear múltipla para analisar o desempenho de corredores de montanha, utilizando dados coletados da plataforma Strava.
\textbf{Metodologia:} Caracteriza-se como um estudo quantitativo, de caráter descritivo e explicativo. Foi realizado um processo de engenharia de variáveis para a criação de métricas de ritmo, estratégia e performance. Subsequentemente, foi aplicada a análise de agrupamento K-Means para a identificação de perfis de corredores e ajustado um modelo de regressão linear múltipla para analisar a relação entre o tempo final e métricas de desempenho e demográficas.
\textbf{Resultados:}
\textbf{Conclusão:} Através deste estudo foram identificados 4 perfis de atletas: Elite, escaladores, especialistas em descidas e os guerreiros resistentes. Ao analisar as relações das variáveis com tempo final de prova conclui-se que o fator mais importante para um bom desempenho foi a constância do ritmo associado ao nível do corredor (clusters), ou seja, fatores demográficos como idade peso e sexo não se mostraram preditores significativos.

\vspace{0.5\baselineskip}
\textbf{Palavras-chave:} Corrida de Montanha, Análise de Desempenho, Análise de Agrupamento, Regressão Linear Múltipla, Strava.
    %====================================================================
    % Resumo em inglês: O resumo em inglês deverá ser escrito dentro 
    % do arquivo abstract.tex.
    %====================================================================
    % Observação: O arquivo abstract.tex já existe dentro da 
    % pasta monografia. O que o aluno precisa fazer é abrí-lo 
    % e preenchê-lo com o texto do abstract. Não esqueça de 
    % de colocar as palavras-chave em inglês. No arquivo 
    % abstract.tex isso poderá ser observado. 
    %====================================================================
    %\chapter*{Abstract}
\thispagestyle{empty}
\label{chap:abstract}

\textbf{Introduction:} Trail running has been gaining prominence among outdoor sports, being influenced by factors such as pace, elevation gain, and gradient. Data from activity tracking platforms allow for the analysis of performance variables, offering a basis to better understand how these elements impact runners' results.
\textbf{Objective:} To apply multiple linear regression models to analyze the performance of trail runners, using data collected from the Strava platform.
\textbf{Methodology:} This is characterized as a quantitative, descriptive, and explanatory study. A feature engineering process was carried out to create metrics for pace, strategy, and performance. Subsequently, K-Means clustering analysis was applied to identify runner profiles, and a multiple linear regression model was fitted to analyze the relationship between the final time and performance and demographic metrics.
\textbf{Results:} The analysis identified a very strong positive correlation ($\rho = 0.94$) between pace variability and the final race time. The final multiple linear regression model showed high explanatory power (Adjusted R² = 0.953). The most significant predictors were pacing consistency—where each second of increase in the standard deviation of pace represented, on average, a penalty of 24.25 seconds on the final time—and the athlete's performance profile (cluster). Demographic factors such as sex and age group lost statistical significance after the inclusion of the clusters in the model.
\textbf{Conclusion:} Through this study, four athlete profiles were identified: Elite, climbers, downhill specialists, and resilient warriors. When analyzing the relationships of the variables with the final race time, it is concluded that the most important factor for good performance was pace consistency associated with the runner's level (clusters), meaning that demographic factors such as age, weight, and sex were not shown to be significant predictors. 

\vspace{0.5\baselineskip}
\textbf{Keywords:} Trail Running, Performance Analysis, Cluster Analysis, Multiple Linear Regression, Strava.

    %====================================================================
    % Comando para inclusão do sumário
    %====================================================================
    %\tableofcontents
    %\thispagestyle{empty}
    %\cleardoublepage
    %====================================================================
    % Comando para inclusão da lista de figuras.
    %==================================================================== 
    % Observação: Caso o trabalho não tenha lista de figuras, comente 
    % as 3 linhas abaixo.
    %====================================================================
    %\pagenumbering{Roman}
    %\listoffigures
    %\cleardoublepage
    %====================================================================
    % Comando para inclusão da lista de tabelas.
    %====================================================================
    % Observação: Caso o trabalho não tenha lista de tabelas, comente 
    % as 2 linhas abaixo.
    %====================================================================
    %\listoftables
    %\cleardoublepage
    %====================================================================
    % Comando para inclusão da lista de abreviações e símbolos.
    %====================================================================
    % Observação: Caso o trabalho não tenha lista de abreviações, 
    % ou lista de símbolos, comente as 2 linhas abaixo.
    %====================================================================
    %\printnomenclature
    %\cleardoublepage
    %====================================================================
    % Capítulos da monografia.
    %====================================================================
    \onehalfspacing
    \clearpage
    \pagenumbering{arabic}
    \pagestyle{cabecalhorodape}

    % Comando para inclusão do sumário
    %====================================================================
    \tableofcontents
    \thispagestyle{empty}
    \cleardoublepage
    %====================================================================
    %====================================================================
    % Introdução.
    %====================================================================
    \newpage
    
\chapter{Introdução}
Os cuidados com a saúde e a prática de esportes têm se tornado, nos últimos anos, hábitos essenciais na vida das pessoas. Dentre os esportes mais praticados destaca-se a corrida \cite{strava_pesquisa}. Em busca de maior contato com a natureza e seus benefícios, corredores têm explorado novas modalidades ao ar livre, entre as quais se destaca o montanhismo, que engloba modalidades como Corrida de Montanha (MC), Trail Running (TR) e Skyrunning. \cite{francagl}.

A obtenção de resultados melhores em corridas depende de alguns fatores, como por exemplo, a capacidade do atleta de controlar a intensidade do esforço ao longo do percurso, adaptando-se aos diferentes tipos de terrenos encontrados. Essa regulação eficiente é determinada pela estratégia de ritmo por quilômetro (\textit{pacing}) adotada, sendo crucial para equilibrar a utilização dos recursos energéticos e retardar a fadiga muscular.  \cite{borgesdl}

Dentre outros fatores, as características ambientais do percurso, como inclinação, tipo de superfície e horário de largada, influenciam significativamente os resultados da prova. O desnível vertical total do percurso, que engloba as alturas acumuladas nas subidas e descidas, é um fator crucial a ser considerado, pois impacta diretamente no esforço requerido e no desempenho do corredor em diferentes trechos da corrida \cite{borgesdl}.

Tais informações e dados dos corredores são encontrados em serviços de rastreamento de exercícios físicos, dentre eles destaca-se o Strava, uma plataforma esportiva que registra os dados dos dispositivos de monitoramento pessoal de cada atleta. Neste serviço, é possível obter informações pessoais dos atletas e fatores como, pacing, frequência cardíaca, ganho de elevação, distância percorrida, tempo de movimentação, dentre outros, captados durante atividades \cite{strava_info}. 

Apesar da vasta literatura sobre fisiologia do exercício, observa-se que muitos estudos sobre desempenho em corrida de montanha se concentram em análises descritivas ou comparações básicas de grupos (e.g., amadores vs. profissionais). Há, contudo, uma lacuna na aplicação de modelos de regressão múltipla que permitam quantificar o impacto simultâneo de variáveis de percurso, como o ganho de elevação acumulado, e de estratégia, como o pacing, no desempenho final. Este trabalho se propõe a preencher essa lacuna, utilizando dados massivos extraídos da plataforma Strava para construir um modelo explicativo robusto.

Ao analisar as informações adquiridas dos corredores na plataforma STRAVA, foi possível obter informações superficiais dos atletas, dessa forma, visando identificar impactos mais profundos, foi adotada uma abordagem estatística utilizando a regressão linear múltipla como principal ferramenta no estudo, com propósito de quantificar o impacto de cada variável no desempenho do corredor, bem como, estimar o valor de uma variável dependente modelando a relação com uma ou mais variáveis independentes \cite{hoffmann}. 

Ao compreender as variáveis que afetam o desempenho, como ritmo, desnível do percurso, é possível identificar estratégias mais eficazes de treinamento e competição, maximizando o potencial de cada corredor assim como contribui para o desenvolvimento de métodos de treinamento mais personalizados e eficazes. 

Quanto ao uso da regressão linear, essa técnica estatística é uma ferramenta poderosa para modelar e compreender as relações entre variáveis independentes e dependentes \cite{hoffmann}.  No contexto da análise do desempenho dos corredores, a regressão linear nos permite quantificar o impacto de cada variável sobre o desempenho geral do corredor, fornecendo uma base sólida para a tomada de decisões informadas no treinamento e na competição. 

Portanto, utilizar a regressão linear neste estudo não apenas nos permite explorar essas relações de forma mais precisa, mas também nos ajuda a identificar padrões e tendências nos dados, fornecendo insights valiosos que podem ser aplicados de maneira prática para melhorar o desempenho e a saúde dos corredores. 

Assim, este estudo não só contribui para o avanço do conhecimento científico na área, mas também tem o potencial de impactar positivamente a comunidade de corredores, auxiliando na melhoria contínua de suas práticas de treinamento e competição.

















    %====================================================================
    % Desenvolvimento.
    %====================================================================
    \newpage
    \section{Justificativa}

Ao observar a lacuna existente na exploração acadêmica ao que se diz respeito a aplicação de técnicas estatísticas envolvendo temas relacionados ao desempenho de atletas, notou-se a viabilidade de elaborar um projeto de pesquisa com ênfase em corredores, sendo eles amadores ou profissionais. 

Entender o desempenho dos corredores é fundamental porque fornece ideias valiosas não apenas para os próprios atletas, mas também para treinadores, profissionais de saúde esportiva e até mesmo pesquisadores da área. Ao compreender as variáveis que afetam o desempenho, como ritmo, desnível do percurso, entre outras, podemos identificar estratégias mais eficazes de treinamento e competição, maximizando o potencial de cada corredor.

Além disso, analisar essas variáveis não apenas fornece uma compreensão mais profunda do desempenho individual, mas também contribui para o desenvolvimento de métodos de treinamento mais personalizados e eficazes. Por exemplo, um corredor pode se beneficiar de ajustes em sua estratégia de pacing com base em suas características físicas e objetivos de desempenho. Da mesma forma, conhecer o impacto do desnível do percurso pode ajudar os atletas a planejar suas estratégias de corrida de forma mais eficiente, adaptando-se melhor às condições específicas da prova.

Quanto ao uso da regressão linear, essa técnica estatística é uma ferramenta poderosa para modelar e compreender as relações entre variáveis independentes e dependentes. No contexto da análise do desempenho dos corredores, a regressão linear nos permite quantificar o impacto de cada variável sobre o desempenho geral do corredor, fornecendo uma base sólida para a tomada de decisões informadas no treinamento e na competição.

Portanto, utilizar a regressão linear neste estudo não apenas nos permite explorar essas relações de forma mais precisa, mas também nos ajuda a identificar padrões e tendências nos dados, fornecendo insights valiosos que podem ser aplicados de maneira prática para melhorar o desempenho e a saúde dos corredores. Assim, este estudo não só contribui para o avanço do conhecimento científico na área, mas também tem o potencial de impactar positivamente a comunidade de corredores, auxiliando na melhoria contínua de suas práticas de treinamento e competição.

    %====================================================================
    % Fundamentação Teórica. 
    %====================================================================
    \newpage
    \chapter{Objetivos}
\label{chap:objetivos}

\section{Objetivo geral:}

O presente estudo teve como objetivo aplicar modelos de regressão linear para analisar o desempenho de corredores de montanha, utilizando dados do Strava. O foco principal foi explorar fatores que influenciam o desempenho final dos atletas em provas de Trail Running (TR).

\section{Objetivos específicos:}

\begin{enumerate}[1.]

\item Utilizar técnicas de web scraping através da linguagem de programação Python, para coletar e organizar uma base de dados contendo informações como tempo, distância, altitude, ritmo, entre outras variávei

\item Explorar e analisar os dados coletados no Strava, a fim de identificar padrões e tendências no desempenho dos corredores, bem como diferenças entre níveis distintos de habilidade;

\item Aplicar modelos de regressão linear para investigar a relação entre variáveis independentes (como desnível, pacing, distância) e dependente (desempenho do corredor);

\item Interpretar os resultados da regressão linear para identificar os principais fatores associados ao desempenho dos corredores;

\item Auxiliar assessorias esportivas e/ou corredores amadores a traçar melhores estratégias para aprimorar o desempenho em uma prova de Trail Running (TR).

\end{enumerate}

    %====================================================================
    % Metodologia.
    %====================================================================
    \newpage
    \chapter{Metodologia}
\label{chap:metodologia}
Este capítulo apresenta a arquitetura metodológica utilizada para a condução desta pesquisa. A estrutura detalha o delineamento da pesquisa, a origem e natureza dos dados, os métodos de coleta e tratamento dos dados, os procedimentos de pré-processamento e agregação, a engenharia de variáveis desenvolvida para aprofundar a análise e, por fim, as técnicas de estatística inferencial empregadas para testar hipóteses e construir um modelo preditivo. Cada etapa é descrita com o objetivo de garantir a transparência e a replicabilidade do estudo. Além de especificar softwares e bibliotecas utilizadas.

\begin{figure}[h!]
    \centering
    \includegraphics[width=0.75\textwidth]{Imagens/fluxo_trabalho.png}
    \caption{Fluxograma do processo de coleta e análise de dados. Fonte: Elaborado pelo autor (2025).}
    \label{fig:fluxograma}
\end{figure}

\section{Delineamento da Pesquisa}

Foi realizado um estudo quantitativo, de caráter descritivo e explicativo. Descritivo por meio da Análise Exploratória de Dados (AED), que visa caracterizar o perfil dos atletas da amostra. Explicativo ao utilizar técnicas de estatística inferencial, como a regressão linear múltipla, para modelar e explicar a relação entre um conjunto de variáveis e o desempenho final dos corredores.

\section{Fonte e Natureza dos Dados}
\label{sec:fonte_dados}

Os dados utilizados neste estudo foram extraídos da plataforma de rastreamento de atividades físicas Strava, referentes às edições de 2022 e 2023 da prova de \emph{trekking} de montanha \textit{La Misión Brasil}, na modalidade de 35 quilômetros. A extração foi realizada por meio de técnicas de \emph{web scraping}, conforme detalhado na Seção \ref{subsec:scraping}.

A natureza do conjunto de dados original era granular, com cada linha representando o desempenho de um único atleta em um quilômetro específico da prova. Essa estrutura, embora detalhada, não era adequada para uma análise centrada no desempenho geral do atleta, necessitando de uma etapa de agregação.

\section{Coleta e Pré-processamento dos Dados}

A obtenção dos dados foi dividida em quatro fases principais: padronização das informações do percurso, extração automatizada de dados da web, limpeza e estruturação final do conjunto de dados.

\subsection{Padronização dos Dados do Percurso via GPX}
\label{subsec:gpx}

A fim de garantir a consistência e a padronização das variáveis geográficas do percurso para todos os atletas, evitando discrepâncias inerentes a diferentes dispositivos de GPS, o ponto de partida foi a análise de um arquivo GPX (GPS Exchange Format) oficial da prova. Este arquivo foi processado para extrair informações de elevação a cada quilômetro, criando um gabarito do percurso com 36 segmentos. Essa abordagem permitiu que variáveis como ganho de elevação positivo e negativo de um determinado quilômetro fossem padronizadas e atribuídas de forma idêntica a todos os corredores da amostra.

\subsection{Extração de Dados via \textit{Web Scraping}}
\label{subsec:scraping}
Para a extração dos dados de desempenho e demográficos dos atletas, foi desenvolvida uma solução de \emph{web scraping} utilizando a linguagem de programação Python e a biblioteca \texttt{Selenium}, que permite a automação de navegadores web. O processo exigiu a autenticação em uma conta de usuário com assinatura paga na plataforma Strava para acessar os dados detalhados. O fluxo de extração foi estruturado da seguinte forma:

\begin{figure}[h!]
    \centering
    \includegraphics[width=0.7\textwidth]{Imagens/fluxo_webscraping.png}
    \caption{Fluxograma do processo de Web Scraping. Fonte: Elaborado pelo autor (2025).}
    \label{fig:fluxograma}
\end{figure}

\begin{enumerate}
    \item \textbf{Coleta de Links:} O ponto de partida foram as páginas de resultados (tabelas de classificação) das edições de 2022 e 2023 da prova. O script navegou por estas páginas e coletou as URLs individuais de cada atividade de atleta registrada.
    
    \item \textbf{Extração de Dados de Performance:} Em um processo iterativo, o script acessava a URL de cada atleta. A prioridade era extrair a tabela de ``Voltas'' (\texttt{laps}), identificada pelo seletor CSS \texttt{'li[data-tracking-element="laps"] a'}, que continha os dados detalhados de tempo a cada quilômetro. Caso esta tabela não estivesse disponível, o script adotava uma abordagem de \emph{fallback}, extraindo os dados da tabela de segmentos principal.
    
    \item \textbf{Extração de Dados Demográficos:} Para obter as variáveis de sexo, faixa etária e faixa de peso, foi empregada uma técnica de filtragem na página principal de resultados. O script aplicava sequencialmente cada filtro (ex: sexo masculino), extraía a lista de nomes dos atletas correspondentes e armazenava essa informação. Posteriormente, esses dados foram unificados com os dados de performance através do nome dos atletas.
    
    \item \textbf{Gestão de Desafios Técnicos:} Durante a execução, foi observado que a plataforma poderia apresentar lentidão no carregamento de elementos. Para contornar essa questão, pausas estratégicas de 5 segundos (\texttt{time.sleep(5)}) foram inseridas no código para garantir que as páginas estivessem completamente carregadas antes da extração, aumentando a robustez do processo.
\end{enumerate}
Importante ressaltar que a coleta foi restrita a dados de perfis e atividades publicamente compartilhados pelos usuários.

\subsection{Limpeza e Estruturação dos Dados}
\label{subsec:limpeza}

Após a extração, foi realizado um rigoroso processo de limpeza e estruturação:
\begin{enumerate}
    \item \textbf{Unificação e Junção:} Os dados de performance dos atletas foram unidos (\emph{join}) aos dados padronizados do percurso (obtidos via GPX) pela coluna indicativa do quilômetro.
    
    \item \textbf{Tratamento de Duplicatas:} Identificou-se que 7 atletas participaram das duas edições da prova. Para garantir a independência das observações, foi mantida apenas a primeira participação de cada um, resultando em 166 atletas únicos nesta fase.
    
    \item \textbf{Padronização de Variáveis:} As colunas de `Tempo` e `Ritmo`, que se apresentavam em formatos distintos, foram unificadas e convertidas para uma unidade padrão de segundos.
    
    \item \textbf{Tratamento de Valores Ausentes:} A estratégia de tratamento de dados faltantes foi definida conforme a variável:
    \begin{itemize}
        \item \textbf{Frequência Cardíaca:} A coluna foi removida do conjunto de dados devido ao alto percentual de valores ausentes.
        \item \textbf{Faixa de Peso:} Com aproximadamente 12,44\% de dados faltantes, optou-se por imputar os valores nulos com a categoria ``Não Informado'', por se tratar de uma variável relevante para a análise.
        \item \textbf{Sexo e Faixa Etária:} Devido ao baixo percentual de ausência (inferior a 10\%), os atletas (linhas) sem essas informações foram removidos do estudo (exclusão \emph{listwise}).
    \end{itemize}
\end{enumerate}
Ao final deste processo, consolidou-se um conjunto de dados final, limpo e estruturado no formato \emph{long}, contendo 109 atletas e 36 observações (quilômetros) para cada um.

\subsection{Agregação dos Dados}
\label{sec:preprocessamento}

Para transformar os dados granulares em um formato analítico focado no atleta, foi realizado um processo de agregação. Utilizando uma operação de agrupamento (\texttt{groupby}) por identificador de atleta, o conjunto de dados foi consolidado, passando de uma estrutura por quilômetro para uma estrutura em que cada linha representava um único atleta. Nesta etapa, foram criadas as variáveis agregadas fundamentais para a análise:
\begin{itemize}
    \item \textbf{\texttt{Tempo\_Final\_seg}:} Variável dependente principal do estudo, calculada como a soma total do tempo de cada quilômetro percorrido pelo atleta.
    \item \textbf{\texttt{Ritmo\_Medio\_seg}:} A média do tempo, em segundos, por quilômetro.
    \item \textbf{\texttt{Variabilidade\_Ritmo\_std}:} O desvio padrão do tempo por quilômetro, servindo como a principal métrica de consistência do ritmo do atleta ao longo da prova.
\end{itemize}

\section{Engenharia de Variáveis}
\label{sec:engenharia_variaveis_final}

Após feita a coleta de dados, assim como seu tratamento, foi realizado a etapa crucial da análise que consistiu na elaboração de novas métricas para capturar nuances da estratégia de prova e da performance dos atletas em diferentes terrenos. 

\subsection{Variáveis de Estratégia de Prova (\textit{Splits})}

Para analisar a distribuição de esforço, a prova foi dividida em duas metades: "Primeira Metade" (quilômetros 1-18) e "Segunda Metade" (quilômetros 19-36). A partir dessa segmentação, foram criadas as variáveis \texttt{Ritmo\_Medio\_Primeira\_Metade} e \texttt{Ritmo\_Medio\_Segunda\_Metade}.

Para uma comparação mais robusta da variação de ritmo, foi criada a variável \texttt{diff\_relativa\_segunda\_primeira\_parte}. A justificativa para o uso de uma métrica relativa, em vez de uma diferença absoluta, reside no fato de que uma mesma variação de tempo (e.g., 30 segundos) possui significados distintos para um atleta de elite e um amador. A normalização pelo ritmo médio do próprio atleta gera uma medida de eficiência de \emph{pacing} mais justa e comparável entre corredores de diferentes níveis de habilidade.

\subsection{Variáveis Baseadas em Terreno (Altimetria)}

Para isolar o desempenho em diferentes inclinações, cada quilômetro do percurso foi classificado como "SUBIDA", "DESCIDA", "PLANO" ou "MISTO". Essa classificação foi realizada por uma função customizada (\texttt{Sob\_Desc}) baseada em regras sobre o desnível positivo e negativo de cada trecho. A partir dessa categorização, foram calculadas as variáveis de ritmo médio para cada tipo de terreno, como \texttt{Ritmo\_Medio\_SUBIDA} e \texttt{Ritmo\_Medio\_DESCIDA}.

\subsection{Índices de Performance Relativa}

Visando quantificar a especialização técnica dos atletas, foram desenvolvidos índices normalizados que comparam o desempenho em terrenos específicos com o desempenho geral do próprio indivíduo.
\begin{itemize}
    \item \textbf{\texttt{indice\_subida}:} Mede o quão mais lento um atleta é na subida em comparação com sua própria média geral.
    \item \textbf{\texttt{indice\_descida}:} Mede o quão mais rápido um atleta é na descida em comparação com sua média geral.
    \item \textbf{\texttt{indice\_descida\_vs\_subida}:} Compara diretamente a performance nos dois terrenos-chave, servindo como uma métrica da capacidade de aceleração relativa do atleta em descidas versus subidas.
\end{itemize}

\section{Análise Estatística}
\label{sec:analise_estatistica}

A análise dos dados foi conduzida por meio de um conjunto de técnicas estatísticas, onde a escolha de cada teste foi rigorosamente justificada pelos pressupostos dos dados.

\subsection{Análise Descritiva}
Inicialmente, foi realizada uma análise descritiva com o uso de medidas de tendência central (média, mediana) e dispersão (desvio padrão) para as variáveis quantitativas, e distribuições de frequência para as variáveis categóricas. Visualizações gráficas como histogramas, diagramas de dispersão e \emph{boxplots} foram utilizadas para compreensão visual dos dados e distribuição das variáveis.

\subsection{Testes de Hipóteses}
\begin{itemize}
    \item \textbf{Comparação entre Sexos:} Para comparar a variável \texttt{Tempo\_Final\_min}, o pressuposto de normalidade foi verificado com o teste de Shapiro-Wilk. A violação do pressuposto (p < 0.05 para ambos os grupos) levou à escolha do teste não-paramétrico de Mann-Whitney U, apropriado para a comparação de duas amostras independentes sem distribuição normal.
    \item \textbf{Comparação entre Faixas Etárias:} Seguindo a mesma lógica, a não-normalidade dos grupos, verificada pelo teste de Shapiro-Wilk, justificou o uso do teste de Kruskal-Wallis. Para investigar as diferenças específicas entre os pares de grupos, foi aplicado o teste \emph{post-hoc} de Dunn com correção de Bonferroni para o controle da taxa de erro do tipo I.
    \item \textbf{Comparação entre Faixas de Peso:} Visando verificar se existe diferença entre as faixas de peso, com a não-normalidade dos grupos, verificada pelo teste de Shapiro-Wilk, justificou o uso do teste de Kruskal-Wallis.
\end{itemize}

\subsection{Análise de Associação}
Para investigar a relação entre performance, \texttt{Tempo\_Final\_min}, e consistência do ritmo, \texttt{Variabilidade\_Ritmo\_min\_std}, a não-normalidade de ambas as variáveis, indicada pelo teste de Shapiro-Wilk, determinou o uso da Correlação de Spearman ($\rho$). Este teste foi escolhido por sua adequação para avaliar avalia a força e a direção das relações monotônicas, independentemente da distribuição dos dados, em contraste com a correlação de Pearson, que pressupõe uma relação linear e normalidade.

\subsection{Análise de Agrupamento}

Com o objetivo de identificar perfis distintos de corredores ("personas") com base em suas características de prova, foi aplicada a técnica de clusterização não-supervisionada K-Means. O propósito desta análise foi segmentar a amostra em grupos intrinsecamente homogêneos e extrinsecamente heterogêneos, revelando diferentes estratégias e especialidades de desempenho dos atletas.

\subsubsection{Pré-processamento das Variáveis para Clusterização}

O algoritmo K-Means agrupa os dados minimizando a variância intra-cluster, um processo que se baseia fundamentalmente no cálculo de distâncias (tipicamente a distância Euclidiana) entre as observações e os centróides dos clusters. Quando as variáveis de entrada (features) possuem escalas e magnitudes muito diferentes — como, por exemplo, uma variável de tempo em segundos (na casa dos milhares) e um índice de performance (na casa de 0.1 a 0.3) — o cálculo da distância pode ser enviesado. A variável com a maior magnitude iria dominar o cálculo, minimizando ou até anulando a contribuição das outras variáveis para a definição dos grupos.\citep{muller2016}.

Para suavizar este efeito e garantir que todas as variáveis tivessem a mesma importância relativa no processo de agrupamento, foi realizado um pré-processamento de padronização dos dados por meio da técnica \textit{StandardScaler}. Este método transforma cada variável para que ela passe a ter uma média igual a zero e um desvio padrão igual a um. A escolha por essa técnica se justifica pois muitos algoritmos de aprendizado de máquina assumem que as características estão centradas em zero e possuem variância na mesma ordem, evitando que uma variável domine a função objetivo do algoritmo \citep{scikitlearn_preprocessing}. A fórmula para a padronização de cada observação (x) em uma variável é dada por:

\begin{equation}
    z = \frac{x - \mu}{\sigma}
    \label{eq:standardscaler}
\end{equation}

Onde:
\begin{itemize}
    \item $z$ é o valor padronizado (ou z-score);
    \item $x$ é o valor original da observação;
    \item $\mu$ é a média de todos os valores da variável (feature);
    \item $\sigma$ é o desvio padrão de todos os valores da variável.
\end{itemize}

Este passo metodológico foi crucial para assegurar que a formação dos clusters fosse resultado dos padrões nos dados, e não de uma distorção causada pelas diferentes escalas das métricas de desempenho.

\subsubsection{Seleção do Número de Clusters (k)}

A determinação do número ideal de agrupamentos ($k$) é um parâmetro fundamental para o algoritmo K-Means. Para esta finalidade, foi empregado o Método do Cotovelo (\textit{Elbow Method})\citep{thorndike1953}. Este método consiste em executar o algoritmo de clusterização para um intervalo de valores de $k$ (e.g., de 1 a 10) e calcular a soma dos quadrados das distâncias intra-cluster (inércia) para cada valor. O número ótimo de clusters é visualmente identificado no ponto do gráfico onde a adição de um novo cluster não resulta em uma diminuição significativa da inércia, formando um "cotovelo" na curva.


\subsection{Modelagem Preditiva}
Para atingir o objetivo principal do trabalho, foi ajustado um modelo de regressão linear múltipla para prever o tempo final dos atletas (variável dependente) a partir de um conjunto de variáveis preditoras (independentes). Para desenvolver um modelo capaz de prever a variável \texttt{Tempo\_Final\_seg}, foi empregada a técnica de Regressão Linear Múltipla, ajustada por Mínimos Quadrados Ordinários (OLS).
\begin{itemize}
    \item \textbf{Refinamento do Modelo:} Partindo de um modelo inicial com múltiplas variáveis, foi aplicado um processo de seleção de variáveis \emph{backward elimination} (eliminação retrógrada), removendo iterativamente os preditores não significativos (com p-valor > 0.05). Adicionalmente, a multicolinearidade entre as variáveis remanescentes foi verificada por meio do Fator de Inflação de Variância (VIF), assegurando a obtenção de um modelo final parcimonioso, robusto e de fácil interpretação.
    \item \textbf{Avaliação do Modelo:} A avaliação do modelo final foi realizada por meio da significância estatística dos coeficientes, da análise do coeficiente de determinação ajustado ($R^2$ ajustado) e da análise gráfica dos resíduos. A análise dos resíduos incluiu a verificação de homocedasticidade, normalidade e independência, garantindo que os pressupostos da regressão linear fossem atendidos.
\end{itemize}

\section{Softwares Utilizados}

Todas as etapas de coleta, tratamento e análise estatística dos dados foram conduzidas na linguagem de programação Python (versão 3.x), com o auxílio de bibliotecas consagradas para manipulação e análise de dados, como Pandas, NumPy, Matplotlib, Seaborn, e para modelagem estatística, como Scikit-learn e Statsmodels.

% --- FIM DO CAPÍTULO DE METODOLOGIA ---

    %====================================================================
    % Resultados Preliminares
    %====================================================================
    \newpage
    \section{Cronograma}

\begin{table}[!h]
\begin{tabular}{|l|l|l|l|l|}
\hline
\textbf{Atividades}                                    & \textbf{Janeiro} & \textbf{Fevereiro} & \textbf{Março} & \textbf{Abril} \\ \hline
Entrega do termo de compromisso                        & X                &                    &                &                \\ \hline
Pesquisa e definição do tema                           & X                &                    &                &                \\ \hline
Revisão bibliográfica                                  & X                & X                  & X              & X              \\ \hline
Elaboração do pré- projeto                             & X                & X                  & X              &                \\ \hline
Entrega do pre-projeto                                 &                  &                    & X              &                \\ \hline
Estudos sobre o assunto                                & X                & X                  & X              & X              \\ \hline
Coleta de Dados (se for o caso)                        &                  & X                  & X              &                \\ \hline
Análise preliminar dos dados (se for o caso)           &                  &                    &                & X              \\ \hline
Preparação do relatório com os resultados preliminares &                  &                    &                & X              \\ \hline
Entrega do relatório com os resultados preliminares    &                  &                    &                & X              \\ \hline
\end{tabular}
\end{table}

    %====================================================================
    % DESCRIÇÃO DE ATIVIDADES E AVALIAÇÃO
    %====================================================================
    \newpage
    \section{Cronograma}

\begin{table}[!h]
\begin{tabular}{|l|l|l|l|l|}
\hline
\textbf{Atividades}                                    & \textbf{Janeiro} & \textbf{Fevereiro} & \textbf{Março} & \textbf{Abril} \\ \hline
Entrega do termo de compromisso                        & X                &                    &                &                \\ \hline
Pesquisa e definição do tema                           & X                &                    &                &                \\ \hline
Revisão bibliográfica                                  & X                & X                  & X              & X              \\ \hline
Elaboração do pré- projeto                             & X                & X                  & X              &                \\ \hline
Entrega do pre-projeto                                 &                  &                    & X              &                \\ \hline
Estudos sobre o assunto                                & X                & X                  & X              & X              \\ \hline
Coleta de Dados (se for o caso)                        &                  & X                  & X              &                \\ \hline
Análise preliminar dos dados (se for o caso)           &                  &                    &                & X              \\ \hline
Preparação do relatório com os resultados preliminares &                  &                    &                & X              \\ \hline
Entrega do relatório com os resultados preliminares    &                  &                    &                & X              \\ \hline
\end{tabular}
\end{table}

    %====================================================================
    % Referências bibliográficas
    %====================================================================
    \newpage
    \def\thispagestyle#1{}
    %\bibliographystyle{babplain-fl}
    \bibliographystyle{plainnat}
    \bibliography{Bibliografia/bibliografia}
    %====================================================================
    % Apêndices da monografia
    % Pode ser que o trabalho tenha vários apêndices. Coloque os 
    % apêndices em uma pasta e dentro da pasta %crie o arquivo tex com o 
    % texto associado ao apêndice. 
    %====================================================================
    %\begin{appendices}
    %\input{apendices/apendice1/apendice1.tex}
    %\input{apendices/apendice2/apendice2.tex}
    %\end{appendices}

    \end{document}
