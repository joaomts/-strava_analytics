\section{Metodologia}

 Neste trabalho de conclusão de curso será realizado um estudo descritivo. Iniciando-se pela extração dos dados de uma prova específica de \textit{skyrunnig} da plataforma de rastreamento de exercícios físicos Strava, através de técnicas de webscraping com a linguagem de programação python. 
 
 A população alvo do estudo consiste em corredores maratonistas de provas de skyrunning. Os dados coletados serão armazenados e organizados em um arquivo Comma Separated Values(CSV), em que as variáveis serão agrupadas por cada corredor participante da prova. Os participantes não serão identificados no conjunto. 

 A análise dos dados coletados se dará por meios de técnicas de estatísticas descritivas buscando compreender características e comportamentos dos corredores da amostra extraída, assim como identificar perfis de corredores através de técnicas de agrupamentos e detecção de perfil. 
 
 Ao ajustar o modelo de regressão linear multivariada utilizando como características o pacing, desnível, frequência cardíaca, entre outras, como variáveis independentes, será possível prever o tempo final de prova do atleta. Tendo como objetivo determinar fatores que influenciarão no desempenho do corredor.

 Para finalizar, serão aplicadas técnicas de avaliação de modelos de regressão linear por meio da análises de resíduos e testes de significância dos coeficientes estimados buscando, assim, entender a relação entre as variáveis independentes e o desempenho dos corredores.


 % --- INÍCIO DO CAPÍTULO DE METODOLOGIA ---

\chapter{Metodologia}
\label{chap:metodologia}

Neste capítulo, são detalhados todos os procedimentos metodológicos empregados para atingir os objetivos propostos por este trabalho. A estrutura abrange o delineamento da pesquisa, a caracterização da população e amostra, os métodos de coleta e tratamento dos dados, a engenharia de variáveis, as técnicas de análise estatística e os softwares utilizados.

\section{Delineamento da Pesquisa}

O presente trabalho configura-se como um estudo quantitativo, de caráter descritivo e explicativo. Descritivo por meio da Análise Exploratória de Dados (AED), que visa caracterizar o perfil dos atletas da amostra. Explicativo ao utilizar técnicas de estatística inferencial, como a regressão linear múltipla, para modelar e explicar a relação entre um conjunto de variáveis e o desempenho final dos corredores.

\section{População e Amostra}

A população do estudo foi composta pelos participantes das edições de 2022 e 2023 da prova de \emph{trekking} de montanha \textit{La Misión Brasil}. O foco da análise recaiu sobre a modalidade de 35 quilômetros, realizada na cidade de Passa Quatro-MG, na região da Serra Fina. O percurso, caracterizado por alta tecnicidade, incluiu trechos de via ferrata e ``escalaminhada'', acumulando aproximadamente 2.791 metros de ganho de elevação positivo.

A amostra foi não-probabilística, por conveniência, composta por atletas cujas atividades relativas à prova estavam publicamente disponíveis na plataforma de rastreamento de exercícios Strava. Após o processo completo de coleta e tratamento dos dados, a amostra final consistiu em \textbf{109 atletas únicos}, sendo 78 do sexo masculino e 31 do sexo feminino.

\section{Coleta e Pré-processamento dos Dados}

A obtenção dos dados foi uma etapa fundamental do estudo, dividida em quatro fases principais: padronização das informações do percurso, extração automatizada de dados da web, limpeza e estruturação final do conjunto de dados.

\subsection{Padronização dos Dados do Percurso via GPX}
\label{subsec:gpx}

A fim de garantir a consistência e a padronização das variáveis geográficas do percurso para todos os atletas, evitando discrepâncias inerentes a diferentes dispositivos de GPS, o ponto de partida foi a análise de um arquivo GPX (GPS Exchange Format) oficial da prova. Este arquivo foi processado para extrair informações de elevação a cada quilômetro, criando um gabarito do percurso com 36 segmentos. Essa abordagem permitiu que variáveis como ganho de elevação positivo e negativo de um determinado quilômetro fossem padronizadas e atribuídas de forma idêntica a todos os corredores da amostra.

\subsection{Extração de Dados via \textit{Web Scraping}}
\label{subsec:scraping}

Para a extração dos dados de desempenho e demográficos dos atletas, foi desenvolvida uma solução de \emph{web scraping} utilizando a linguagem de programação Python e a biblioteca \texttt{Selenium}, que permite a automação de navegadores web. O processo exigiu a autenticação em uma conta de usuário com assinatura paga na plataforma Strava para acessar os dados detalhados.

O fluxo de extração foi estruturado da seguinte forma:
\begin{enumerate}
    \item \textbf{Coleta de Links:} O ponto de partida foram as páginas de resultados (tabelas de classificação) das edições de 2022 e 2023 da prova. O script navegou por estas páginas e coletou as URLs individuais de cada atividade de atleta registrada.
    
    \item \textbf{Extração de Dados de Performance:} Em um processo iterativo, o script acessava a URL de cada atleta. A prioridade era extrair a tabela de ``Voltas'' (\texttt{laps}), identificada pelo seletor CSS \texttt{'li[data-tracking-element="laps"] a'}, que continha os dados detalhados de tempo a cada quilômetro. Caso esta tabela não estivesse disponível, o script adotava uma abordagem de \emph{fallback}, extraindo os dados da tabela de segmentos principal.
    
    \item \textbf{Extração de Dados Demográficos:} Para obter as variáveis de sexo, faixa etária e faixa de peso, foi empregada uma técnica de filtragem na página principal de resultados. O script aplicava sequencialmente cada filtro (ex: sexo masculino), extraía a lista de nomes dos atletas correspondentes e armazenava essa informação. Posteriormente, esses dados foram unificados com os dados de performance através do nome dos atletas.
    
    \item \textbf{Gestão de Desafios Técnicos:} Durante a execução, foi observado que a plataforma poderia apresentar lentidão no carregamento de elementos. Para contornar essa questão, pausas estratégicas de 5 segundos (\texttt{time.sleep(5)}) foram inseridas no código para garantir que as páginas estivessem completamente carregadas antes da extração, aumentando a robustez do processo.
\end{enumerate}
A coleta foi restrita a dados de perfis e atividades publicamente compartilhados pelos usuários.

\subsection{Limpeza e Estruturação dos Dados}
\label{subsec:limpeza}

Após a extração, foi realizado um rigoroso processo de limpeza e estruturação:
\begin{enumerate}
    \item \textbf{Unificação e Junção:} Os dados de performance dos atletas foram unidos (\emph{join}) aos dados padronizados do percurso (obtidos via GPX) pela coluna indicativa do quilômetro.
    
    \item \textbf{Tratamento de Duplicatas:} Identificou-se que 7 atletas participaram das duas edições da prova. Para garantir a independência das observações, foi mantida apenas a primeira participação de cada um, resultando em 166 atletas únicos nesta fase.
    
    \item \textbf{Padronização de Variáveis:} As colunas de `Tempo` e `Ritmo`, que se apresentavam em formatos distintos, foram unificadas e convertidas para uma unidade padrão de segundos.
    
    \item \textbf{Tratamento de Valores Ausentes:} A estratégia de tratamento de dados faltantes foi definida conforme a variável:
    \begin{itemize}
        \item \textbf{Frequência Cardíaca:} A coluna foi removida do conjunto de dados devido ao alto percentual de valores ausentes.
        \item \textbf{Faixa de Peso:} Com aproximadamente 12,44\% de dados faltantes, optou-se por imputar os valores nulos com a categoria ``Não Informado'', por se tratar de uma variável relevante para a análise.
        \item \textbf{Sexo e Faixa Etária:} Devido ao baixo percentual de ausência (inferior a 10\%), os atletas (linhas) sem essas informações foram removidos do estudo (exclusão \emph{listwise}).
    \end{itemize}
\end{enumerate}
Ao final deste processo, consolidou-se um conjunto de dados final, limpo e estruturado no formato \emph{long}, contendo 109 atletas e 36 observações (quilômetros) para cada um.

\section{Engenharia de Variáveis}

Para aprofundar a análise, variáveis originais foram transformadas e novas variáveis foram criadas. As principais foram:
\begin{itemize}
    \item \textbf{Variabilidade do Ritmo:} Calculada como o desvio padrão do \emph{pace} (em segundos por quilômetro) para cada atleta, servindo como uma medida da consistência do ritmo ao longo da prova.
    
    \item \textbf{Faixas Etárias e de Peso:} Foram utilizadas as variáveis categóricas ordinais extraídas do Strava. A definição das faixas seguiu a própria categorização utilizada pela plataforma.
\end{itemize}




\section{Análise Estatística}
% O texto desta seção foi mantido da nossa discussão anterior.
% Sinta-se à vontade para alterá-lo se achar necessário.
\subsection{Análise Exploratória de Dados (AED)}
Inicialmente, foi realizada uma AED para resumir e visualizar as principais características dos dados. Foram calculadas medidas de tendência central (média, mediana) e de dispersão (desvio padrão, amplitude interquartil) para as variáveis quantitativas, e distribuições de frequência para as variáveis categóricas. Gráficos como histogramas, boxplots e diagramas de dispersão foram utilizados para a compreensão visual dos dados.

\subsection{Testes de Hipóteses}
Para comparar o desempenho entre diferentes grupos de corredores, foram aplicados testes de hipóteses não-paramétricos, devido à natureza da distribuição de algumas variáveis. O teste de Mann-Whitney foi utilizado para comparar o tempo final entre os sexos, enquanto o teste de Kruskal-Wallis foi empregado para comparar o desempenho entre as diferentes faixas etárias e de peso.

\subsection{Análise de Correlação}
Para investigar a associação entre a performance e a consistência do ritmo, foi utilizada a Correlação de Spearman ($\rho$), um teste não-paramétrico que avalia a força e a direção da relação monotônica entre o tempo final e a variabilidade do ritmo.

\subsection{Análise de Agrupamento (\textit{Cluster})}
Com o objetivo de identificar perfis distintos de corredores com base em suas características de prova, foi aplicada a técnica de clusterização não-supervisionada K-Means. A seleção do número ótimo de \emph{clusters} (k) foi realizada por meio do método do cotovelo (\textit{Elbow Method}).

\subsection{Modelo de Regressão Linear Múltipla}
Para atingir o objetivo principal do trabalho, foi ajustado um modelo de regressão linear múltipla para prever o tempo final dos atletas (variável dependente) a partir de um conjunto de variáveis preditoras (independentes). Partindo de um modelo inicial completo, foi aplicado um procedimento de seleção de variáveis para obter um modelo final parcimonioso e de boa interpretação. A significância estatística dos coeficientes, a análise do coeficiente de determinação ajustado ($R^2$ ajustado) e a análise gráfica dos resíduos foram utilizadas para avaliar a qualidade e a adequação do modelo final.

\section{Softwares Utilizados}

Todas as etapas de coleta, tratamento e análise estatística dos dados foram conduzidas na linguagem de programação Python (versão 3.x), com o auxílio de bibliotecas consagradas para manipulação e análise de dados, como Pandas, NumPy, Matplotlib, Seaborn, e para modelagem estatística, como Scikit-learn e Statsmodels.

% --- FIM DO CAPÍTULO DE METODOLOGIA ---


 
 
 
