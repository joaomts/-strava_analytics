
\section{Introdução}

Os cuidados com a saúde e a prática de esportes têm se tornado, nos últimos anos, hábitos essenciais na vida das pessoas. Dentre os esportes mais praticados destaca-se a corrida (STRAVA, 2023)\cite{strava_pesquisa}. Buscando maior contato com a natureza e seus benefícios, corredores têm explorado novas modalidades ao ar livre, dentre elas está a prática de montanhismo (FRANÇA et al, 2021)\cite{francagl}, que se insere no mundo esportivo com as seguintes modalidades: Corrida de Montanha(MC), o Trail Running(TR), o Skyrunnig, entre outros. 

A obtenção de melhores resultados em corridas depende de alguns fatores, como por exemplo, a capacidade do atleta de controlar a intensidade do esforço ao longo do percurso, adaptando-se aos diferentes tipos de terrenos encontrados. Essa regulação eficiente é determinada pela estratégia de ritmo por quilômetro (pacing) adotada, sendo crucial para equilibrar a utilização dos recursos energéticos e retardar a fadiga muscular. 

Dentre outros fatores, as características ambientais do percurso, como inclinação, tipo de superfície e horário de largada, influenciam significativamente nos resultados da prova. O desnível vertical total do percurso, que engloba as alturas acumuladas nas subidas e descidas, é um fator crucial a ser considerado, pois impacta diretamente no esforço requerido e no desempenho do corredor em diferentes trechos da corrida \cite{borgesdl}.

Tais informações e dados dos corredores são encontradas em serviços de rastreamento de exercícios físicos, dentre eles destaca-se o Strava, uma plataforma esportiva que registra os dados dos dispositivos de monitoramento pessoal de cada atleta. Neste serviço, é possível obter informações pessoais dos atletas e fatores como, pacing, frequência cardíaca, ganho de elevação, distância percorrida, tempo de movimentação, dentre outros, captados durante atividades (STRAVA, 2024) \cite{strava_info}.

Previamente, será possível obter conclusões apenas analisando as informações adquiridas dos corredores, entretanto, visando identificar impactos mais profundos, será adotada uma abordagem estatística utilizando a regressão linear múltipla como principal ferramenta no estudo, com propósito de quantificar o impacto de cada variável no desempenho do corredor, bem como, estimar o valor de uma variável dependente modelando a relação com uma ou mais variáveis independentes\cite{hoffmann}.

Devido a discrepância entre o alto crescimento de adeptos à modalidade de TR e a baixa quantidade de estudos envolvendo esse estilo de corrida, considerando aspectos de provas reais, fica evidente que o tema tem alto potencial para ser abordado. Em síntese, o objetivo deste trabalho é analisar fatores que influenciam no desempenho final atletas em prova de TR.




















